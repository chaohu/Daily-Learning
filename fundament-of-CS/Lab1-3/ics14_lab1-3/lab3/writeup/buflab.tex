
\documentclass[11pt]{article}

\usepackage{times}
\usepackage{alltt}
\usepackage{url}


%% Page layout
\oddsidemargin 0pt
\evensidemargin 0pt
\textheight 600pt
\textwidth 469pt
\setlength{\parindent}{0em}
\setlength{\parskip}{1ex}

%% ccode environment- for displaying formatted C code (c2tex) 
\newenvironment{ccode}%
{\small}%
{}

%% tty - for displaying TTY input and output
\newenvironment{tty}%
{\small\begin{alltt}}%
{\end{alltt}}




\begin{document}

\title{15-213, Fall 20xx\\
Lab Assignment L3: The Buffer Bomb\\
Assigned: XXX, Due: XXX\\
Last Possible Time to Turn in: XXX
}

\author{}
\date{}

\maketitle



Harry Bovik (bovik@cs.cmu.edu) is the lead person for
this assignment.


\section*{Introduction}

This assignment will help you develop a detailed understanding of
IA-32 calling conventions and stack organization.  It involves
applying a series of {\em buffer overflow attacks} on an executable
file {\tt bufbomb} in the lab directory.

{\bf Note:} In this lab, you will gain firsthand experience with one
of the methods commonly used to exploit security weaknesses in
operating systems and network servers.  Our purpose is to help you
learn about the runtime operation of programs and to understand the
nature of this form of security weakness so that you can avoid it when
you write system code.  We do not condone the use of this or any other
form of attack to gain unauthorized access to any system resources.
There are criminal statutes governing such activities. 


\section*{Logistics}

As usual, this is an individual project.

We generated the lab using {\tt gcc}'s {\tt -m32} flag, so all code
produced by the compiler follows IA-32 rules, even if the host is an
x86-64 system. This should be enough to convince you that the compiler
can use any calling convention it wants, so long as it's consistent.


\section*{Hand Out Instructions}

You can obtain your buffer bomb by pointing your Web browser at:

\begin{verbatim}
    http://$Buflab::SERVER_NAME:18213/
\end{verbatim}

The server will return a \texttt{tar} file called
\texttt{buflab-handout.tar} to your browser.  Start by copying
\texttt{buflab-handout.tar} to a (protected) directory in which you
plan to do your work.  Then give the command
``\verb@tar xvf buflab-handout.tar@''.  This will create a directory
called \texttt{buflab-handout} containing the following three
executable files:
\begin{description}
\item {\bf bufbomb}: The buffer bomb program you will attack.
\item {\bf makecookie}: Generates a ``cookie'' based on your userid.
\item {\bf hex2raw}: A utility to help convert between string formats.
\end{description}

In the following instructions, we will assume that you have copied the
three programs to a protected local directory, and that you are
executing them in that local directory.


\section*{Userids and Cookies}

Phases of this lab will require a slightly different solution from
each student. The correct solution will be based on your userid.

A {\em cookie} is a string of eight hexadecimal digits that is (with
high probability) unique to your userid.  You can generate your cookie
with the {\tt makecookie} program giving your userid as the
argument.  For example:
\begin{tty}
unix>{\em ./makecookie bovik}
0x1005b2b7
\end{tty}

In four of your five buffer attacks, your objective will be to make
your cookie show up in places where it ordinarily would not.


\section*{The {\sc bufbomb} Program}

The {\sc bufbomb} program reads a string from standard input. It does
so with the function {\tt getbuf} defined below:

\begin{ccode}
\begin{alltt}
{\scriptsize   1} /* Buffer size for getbuf */
{\scriptsize   2} #define NORMAL_BUFFER_SIZE 32
{\scriptsize   3} 
{\scriptsize   4} int getbuf()
{\scriptsize   5} \verb:{:
{\scriptsize   6}     char buf[NORMAL_BUFFER_SIZE];
{\scriptsize   7}     Gets(buf);
{\scriptsize   8}     return 1;
{\scriptsize   9} \verb:}:
\end{alltt}

\end{ccode}

The function {\tt Gets} is similar to the standard library function
{\tt gets}---it reads a string from standard input (terminated by
`\verb@\n@' or end-of-file) and stores it (along with a null
terminator) at the specified destination.  In this code, you can see
that the destination is an array {\tt buf} having sufficient space for
32 characters.

{\tt Gets} (and {\tt gets}) grabs a string off the input stream and
stores it into its destination address (in this case {\tt
buf}). However, {\tt Gets()} has no way of determining whether {\tt
buf} is large enough to store the whole input.  It simply copies the
entire input string, possibly overrunning the bounds of the storage
allocated at the destination.

If the string typed by the user to {\tt getbuf} is no more than 31
characters long, it is clear that {\tt getbuf} will return 1, as shown
by the following execution example:

\begin{tty}
    unix>{\em ./bufbomb -u bovik}
    Type string:{\em I love 15-213.}
    Dud: getbuf returned 0x1
\end{tty}

Typically an error occurs if we type a longer string:

\begin{tty}
    unix>{\em ./bufbomb -u bovik}
    Type string:{\em It is easier to love this class when you are a TA.}
    Ouch!: You caused a segmentation fault!
\end{tty}

As the error message indicates, overrunning the buffer typically
causes the program state to be corrupted, leading to a memory access
error.  Your task is to be more clever with the strings you feed {\sc
bufbomb} so that it does more interesting things.  These are called
{\em exploit} strings.

{\sc Bufbomb} takes several different command line arguments:
\begin{description}
\item[{\tt -u} {\it userid}:] Operate the bomb for the indicated userid.
You should always provide this argument for several reasons:


\begin{itemize}

\item It is required to submit your successful attacks to the grading server.

\item {\sc bufbomb} determines the cookie you will be using based on
your userid, as does the program {\sc makecookie}.

\item We have built features into {\sc bufbomb} so that some of the
key stack addresses you will need to use depend on your userid's
cookie.

\end{itemize}

\item[{\tt -h}:] Print list of possible command line arguments.
\item[{\tt -n}:] Operate in ``Nitro'' mode, as is used in Level 4 below.
\item[{\tt -s}:] Submit your solution exploit string to the grading server.

\end{description}

At this point, you should think about the x86 stack structure a bit
and figure out what entries of the stack you will be targeting. You
may also want to think about {\em exactly} why the last example
created a segmentation fault, although this is less clear.

Your exploit strings will typically contain byte values that do not
correspond to the ASCII values for printing characters.  The program
{\sc hex2raw} can help you generate these {\em raw} strings.  It takes
as input a {\em hex-formatted} string.  In this format, each byte
value is represented by two hex digits.  For example, the string
``{\tt 012345}'' could be entered in hex format as ``{\tt 30 31 32 33
34 35}.'' (Recall that the ASCII code for decimal digit $x$ is {\tt
0x3}$x$.)

The hex characters you pass {\sc hex2raw} should be separated by
whitespace (blanks or newlines). I recommend separating different
parts of your exploit string with newlines while you're working on
it. {\sc hex2raw} also supports C-style block comments, so you can
mark off sections of your exploit string. For example:

\begin{tty}
bf 66 7b 32 78 /* mov    $0x78327b66,%edi */
\end{tty}
%% $ (De-confuse text editor's autoformatting)

Be sure to leave space around both the starting and ending comment strings (
`\verb@/*@', `\verb@*/@') so they will be properly ignored.


If you generate a hex-formatted exploit string in the file {\tt
exploit.txt}, you can apply the raw string to {\sc bufbomb} in several
different ways:
\begin{enumerate}
\item You can set up a series of pipes to pass the string through {\sc hex2raw}.
\begin{tty}
    unix>{\em cat exploit.txt | ./hex2raw | ./bufbomb -u bovik}
\end{tty}
\item
You can store the raw string in a file and use I/O redirection to
supply it to {\sc bufbomb}:
\begin{tty}
    unix>{\em ./hex2raw < exploit.txt > exploit-raw.txt}
    unix>{\em ./bufbomb -u bovik < exploit-raw.txt}
\end{tty}
This approach can also be used when running {\sc bufbomb} from within
{\sc gdb}:
\begin{tty}
    unix>{\em gdb bufbomb}
    (gdb){\em run -u bovik < exploit-raw.txt}
\end{tty}
\end{enumerate}


\textbf{Important points:}

\begin{itemize}

\item Your exploit string must not contain byte value {\tt 0x0A} at
any intermediate position, since this is the ASCII code for newline
(`\verb@\n@').  When {\tt Gets} encounters this byte, it will assume
you intended to terminate the string.

\item {\sc hex2raw} expects two-digit hex values separated by a 
whitespace. So if you want to create a byte with a hex value of 0, 
you need to specify 00.  To create the word {\tt 0xDEADBEEF} you 
should pass DE AD BE EF to {\sc hex2raw}.

\end{itemize}

When you have correctly solved one of the levels, say level 0:

\begin{tty}
    ../hex2raw < smoke-bovik.txt | ../bufbomb -u bovik 
    Userid: bovik
    Cookie: 0x1005b2b7
    Type string:Smoke!: You called smoke()
    VALID
    NICE JOB!
\end{tty}
then you can submit your 
solution to the grading server using the \texttt{-s} option:
\begin{tty}
    ./hex2raw < smoke-bovik.txt | ./bufbomb -u bovik -s
    Userid: bovik
    Cookie: 0x1005b2b7
    Type string:Smoke!: You called smoke()
    VALID
    Sent exploit string to server to be validated.
    NICE JOB!
\end{tty}
The server will test your exploit string to make sure it really works,
and it will update the Buffer Lab scoreboard page indicating that your
userid (listed by your cookie for anonymity) has completed this level.

You can view the scoreboard by pointing your Web browser at 
\begin{verbatim}
    http://$Buflab::SERVER_NAME:18213/scoreboard
\end{verbatim}

Unlike the Bomb Lab, there is no penalty for making mistakes in this
lab.  Feel free to fire away at {\sc bufbomb} with any string you
like.  Of course, you shouldn't brute force this lab either, since it
would take longer than you have to do the assignment.

IMPORTANT NOTE: You can work on your buffer bomb on any Linux machine, but 
in order to submit your solution, you will need to be running on one of the 
following machines:
\begin{verbatim}
INSTRUCTOR: Insert the list of the legal domain names that you 
established in buflab/src/config.h.
\end{verbatim}

\section*{Level 0: Candle (10 pts)}

The function {\tt getbuf} is called within {\sc bufbomb} by a function
{\tt test} having the following C code:

\begin{ccode}
\input boom-c
\end{ccode}

When {\tt getbuf} executes its return statement (line 5 of {\tt
getbuf}), the program ordinarily resumes execution within function
{\tt test} (at line 7 of this function). We want to change this behavior.
Within the file {\tt bufbomb}, there is a function {\tt smoke} having
the following C code:

\begin{ccode}
\input smoke-c
\end{ccode}

Your task is to get {\sc bufbomb} to execute the code for {\tt smoke}
when {\tt getbuf} executes its return statement, rather than returning
to {\tt test}.  Note that your exploit string may also corrupt parts
of the stack not directly related to this stage, but this will not
cause a problem, since {\tt smoke} causes the program to exit
directly.


{\bf Some Advice}:
\begin{itemize}

\item All the information you need to devise your exploit string for
this level can be determined by examining a disassembled version of
{\sc bufbomb}. Use {\tt objdump -d} to get this dissembled version.

\item
Be careful about byte ordering.  

\item
You might want to use {\sc gdb} to step the program through the last
few instructions of {\tt getbuf} to make sure it is doing the right
thing.

\item
The placement of {\tt buf} within the stack frame for {\tt getbuf}
depends on which version of {\sc gcc} was used to compile {\tt
bufbomb}, so you will have to read some assembly to figure out its
true location. 
\end{itemize}


\section*{Level 1: Sparkler (10 pts)}

Within the file {\tt bufbomb} there is also a function {\tt fizz}
having the following C code:

\begin{ccode}
\input fizz-c
\end{ccode}

Similar to Level 0, your task is to get {\sc bufbomb} to execute the
code for {\tt fizz} rather than returning to {\tt test}.  In this
case, however, you must make it appear to {\tt fizz} as if you have
passed your cookie as its argument. How can you do this?


{\bf Some Advice}:
\begin{itemize}

\item Note that the program won't really call {\tt fizz}---it will
simply execute its code. This has important implications for where on
the stack you want to place your cookie.

\end{itemize}

\section*{Level 2: Firecracker (15 pts)}

A much more sophisticated form of buffer attack involves supplying a
string that encodes actual machine instructions.  The exploit string
then overwrites the return pointer with the starting address of these
instructions on the stack.  When the calling function (in this case
{\tt getbuf}) executes its {\tt ret} instruction, the program will
start executing the instructions on the stack rather than returning.
With this form of attack, you can get the program to do almost
anything.  The code you place on the stack is called the {\em exploit}
code.  This style of attack is tricky, though, because you must get
machine code onto the stack and set the return pointer to the start of
this code.

Within the file {\tt bufbomb} there is a function {\tt bang}
having the following C code:

\begin{ccode}
\input bang-c
\end{ccode}

Similar to Levels 0 and 1, your task is to get {\sc bufbomb} to
execute the code for {\tt bang} rather than returning to {\tt test}.
Before this, however, you must set global variable \verb@global_value@
to your userid's cookie.  Your exploit code should set
\verb@global_value@, push the address of \texttt{bang} on the stack,
and then execute a \texttt{ret} instruction to cause a jump to the
code for \texttt{bang}.

{\bf Some Advice}:
\begin{itemize}

\item
You can use {\sc gdb} to get the information you need to construct
your exploit string.  Set a breakpoint within {\tt getbuf} and run to
this breakpoint.  Determine parameters such as the address of
\verb@global_value@ and the location of the buffer.

\item
Determining the byte encoding of instruction sequences by hand is
tedious and prone to errors.  You can let tools do all of the work by
writing an assembly code file containing the instructions and data you
want to put on the stack.  Assemble this file with {\tt gcc -m32 -c} and
disassemble it with {\tt objdump -d}.  You should be able to get the
exact byte sequence that you will type at the prompt.  (A brief
example of how to do this is included at the end of this writeup.)

\item
Keep in mind that your exploit string depends on your machine, your
compiler, and even your userid's cookie.  Do all of your work on one
of the machines assigned by your instructor, and make sure you include
the proper userid on the command line to {\sc bufbomb}.

\item
Watch your use of address modes when writing assembly code.  Note that
\verb@movl $0x4, %eax@
%% $ (deconfuse text editor)
moves the {\em value} \texttt{0x00000004} into
register \verb@%eax@; whereas \verb@movl 0x4, %eax@ moves the value
{\em at} memory location \texttt{0x00000004} into \verb@%eax@. Since
that memory location is usually undefined, the second instruction will
cause a segfault!

\item
Do not attempt to use either a \texttt{jmp} or a \texttt{call}
instruction to jump to the code for \texttt{bang}.  These instructions
uses PC-relative addressing, which is very tricky to set up correctly.
Instead, push an address on the stack and use the \texttt{ret}
instruction.

\end{itemize}

\section*{Level 3: Dynamite (20 pts)}

Our preceding attacks have all caused the program to jump to the code
for some other function, which then causes the program to exit.  As a
result, it was acceptable to use exploit strings that corrupt the
stack, overwriting saved values.

The most sophisticated form of buffer overflow attack causes the
program to execute some exploit code that changes the program's
register/memory state, but makes the program return to the original
calling function (\texttt{test} in this case).  The calling function
is oblivious to the attack.  This style of attack is tricky, though,
since you must: 1) get machine code onto the stack, 2) set the return
pointer to the start of this code, and 3) undo any corruptions made to
the stack state.

Your job for this level is to supply an exploit string that will cause
{\tt getbuf} to return your cookie back to \texttt{test}, rather than
the value 1.  You can see in the code for {\tt test} that this will
cause the program to go ``{\tt Boom!}.''  Your exploit code should set
your cookie as the return value, restore any corrupted state, push the
correct return location on the stack, and execute a {\tt ret}
instruction to really return to {\tt test}.

{\bf Some Advice}:
\begin{itemize}

\item 
You can use {\sc gdb} to get the information you need to construct
your exploit string.  Set a breakpoint within {\tt getbuf} and run to
this breakpoint.  Determine parameters such as the saved return
address.

\item
Determining the byte encoding of instruction sequences by hand is
tedious and prone to errors.  You can let tools do all of the work by
writing an assembly code file containing the instructions and data you
want to put on the stack.  Assemble this file with {\sc gcc} and
disassemble it with {\sc objdump}.  You should be able to get the
exact byte sequence that you will type at the prompt.  
(A brief example of how to do this is included at the end of this writeup.)

\item
Keep in mind that your exploit string depends on your machine, your
compiler, and even your userid's cookie.  Do all of your work on the
machines assigned by your instructor, and make sure you include the
proper userid on the command line to {\sc bufbomb}.
\end{itemize}

Once you complete this level, pause to reflect on what you have
accomplished.  You caused a program to execute machine code of your
own design.  You have done so in a sufficiently stealthy way that the
program did not realize that anything was amiss.


\section*{Level 4: Nitroglycerin (10 pts)}

Please note: You'll need to use the ``{\tt -n},'' command-line flag in
order to run this stage.

From one run to another, especially by different users, the exact
stack positions used by a given procedure will vary.  One reason for
this variation is that the values of all environment variables are
placed near the base of the stack when a program starts executing.
Environment variables are stored as strings, requiring different
amounts of storage depending on their values.  Thus, the stack space
allocated for a given user depends on the settings of his or her
environment variables.  Stack positions also differ when running a
program under {\sc gdb}, since {\sc gdb} uses stack space for some of its
own state.

In the code that calls {\tt getbuf}, we have incorporated features
that stabilize the stack, so that the position of {\tt getbuf}'s stack
frame will be consistent between runs.  This made it possible for you
to write an exploit string knowing the exact starting address of {\tt
buf}.  If you tried to use such an exploit on a normal program, you
would find that it works some times, but it causes segmentation faults
at other times.  Hence the name ``dynamite''---an explosive developed
by Alfred Nobel that contains stabilizing elements to make it less
prone to unexpected explosions.

For this level, we have gone the opposite direction, making the stack
positions even less stable than they normally are.  Hence the name
``nitroglycerin''---an explosive that is notoriously unstable.

When you run {\sc bufbomb} with the command line flag ``{\tt -n},'' it
will run in ``Nitro'' mode.  Rather than calling the function {\tt
getbuf}, the program calls a slightly different function {\tt
getbufn}:

\begin{ccode}
\input kaboom-c
\end{ccode}

This function is similar to {\tt getbuf}, except that it has a buffer
of 512 characters.  You will need this additional space to create a
reliable exploit.  The code that calls {\tt getbufn} first allocates a
random amount of storage on the stack,
such that if you
sample the value of \verb@%ebp@ during two successive executions of
{\tt getbufn}, you would find they differ by as much as $\pm 240$.

In addition, when run in Nitro mode, {\sc bufbomb} requires you to
supply your string 5 times, and it will execute {\tt getbufn} 5 times,
each with a different stack offset.  Your exploit string must make it
return your cookie each of these times.

Your task is identical to the task for the Dynamite level. Once again,
your job for this level is to supply an exploit string that will cause
\texttt{getbufn} to return your cookie back to test, rather than the value 1.
You can see in the code for test that this will cause the program to go
``{\tt KABOOM!}.'' Your exploit code should set your cookie as the return
value, restore any corrupted state, push the correct return location on
the stack, and execute a \verb@ret@ instruction to really return to
{\tt testn}.

{\bf Some Advice}:
\begin{itemize}
\item You can use the program {\sc hex2raw} to send multiple copies
of your exploit string.  If you have a single copy in the file {\tt
exploit.txt}, then you can use the following command:

\begin{tty}
    unix>{\em cat exploit.txt | ./hex2raw -n | ./bufbomb -n -u bovik}
\end{tty}

You must use the same string for all 5 executions of {\tt getbufn}.
Otherwise it will fail the testing code used by our grading server.

\item
The trick is to make use of the {\tt nop} instruction.  It is encoded
with a single byte (code {\tt 0x90}). It may be useful to read about
"nop sleds" on page 262 of the CS:APP2e textbook.

\end{itemize}


\section*{Logistical Notes}

Handin occurs to the grading server whenever you correctly solve a
level {\em and} use the \texttt{-s} option.  Upon receiving your
solution, the server will validate your string and update the Buffer
Lab scoreboard Web page, which you can view by pointing your Web
browser at
\begin{verbatim}
    http://$Buflab::SERVER_NAME:18213/scoreboard
\end{verbatim}

You should be sure to check this page after your submission to
make sure your string has been validated. (If you really solved the
level, your string {\em should} be valid.)

Note that each level is graded individually.  You do not need to do
them in the specified order, but you will get credit only for the
levels for which the server receives a valid message. You can check
the Buffer Lab scoreboard to see how far you've gotten. 

The grading server creates the scoreboard by using the latest results
it has for each phase. 

Good luck and have fun!

\section*{Generating Byte Codes}

Using {\sc gcc} as an assembler and {\sc objdump} as a disassembler
makes it convenient to generate the byte codes for instruction
sequences.  For example, suppose we write a file \verb@example.S@
containing the following assembly code:
\begin{tty}
\input{example.S}
\end{tty}

The code can contain a mixture of instructions and data.  Anything to
the right of a `\verb@#@' character is a comment.  

We can now assemble and disassemble this file:
\begin{tty}
    unix>{\em gcc -m32 -c example.S}
    unix>{\em objdump -d example.o > example.d}
\end{tty}
The generated file {\tt example.d} contains the following lines
\begin{ccode}
    \begin{alltt}
   0:	68 ef cd ab 00       	push   $0xabcdef
   5:	83 c0 11             	add    $0x11,%eax
   8:	98                   	cwtl   
   9:	ba                   	.byte 0xba
   a:	dc fe                	fdivr  %st,%st(6)
\end{alltt}

\end{ccode}
Each line shows a single instruction.  The number on the left
indicates the starting address (starting with 0), while the hex digits
after the `\verb@:@' character indicate the byte codes for the
instruction.  Thus, we can see that the instruction {\tt push
\$0xABCDEF} has hex-formatted byte code {\tt 68 ef cd ab 00}.

Starting at address 8, the disassembler gets confused.  It tries to
interpret the bytes in the file {\tt example.o} as instructions, but
these bytes actually correspond to data.  Note, however, that if we
read off the 4 bytes starting at address 8 we get: {\tt 98 ba dc fe}.
This is a byte-reversed version of the data word {\tt 0xFEDCBA98}.
This byte reversal represents the proper way to supply the bytes as a
string, since a little endian machine lists the least significant byte
first.  

Finally, we can read off the byte sequence for our code as:
\begin{tty}
\input{example.txt}
\end{tty}
This string can then be passed through {\sc hex2raw} to generate a
proper input string we can give to {\sc bufbomb}.  Alternatively, we
can edit example.d to look like this:
\begin{ccode}
\begin{alltt}
68 ef cd ab 00 /* push   $0xabcdef */{\em\scriptsize }
83 c0 11 /* add    $0x11,%eax */{\em\scriptsize }
98 
ba dc fe 
\end{alltt}
\end{ccode}
which is also a valid input we can pass through {\sc hex2raw} before
sending to {\sc bufbomb}.
\end{document}

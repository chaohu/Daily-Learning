\documentclass[11pt]{article}

\usepackage{times}


%% Page layout
\oddsidemargin 0pt
\evensidemargin 0pt
\textheight 600pt
\textwidth 469pt
\setlength{\parindent}{0em}
\setlength{\parskip}{1ex}

\begin{document}

\title{15-213, Fall 20xx\\
Data Lab: Manipulating Bits\\
Assigned: Aug.~30, Due: Wed., Sept.~12, 11:59PM
}

\author{}
\date{}

\maketitle

Harry Bovik ({\tt bovik@cs.cmu.edu}) is the lead person for
this assignment.

\section{Introduction}
The purpose of this assignment is to become more familiar with
bit-level representations of integers and floating point numbers.
You'll do this by solving a series of programming ``puzzles.'' Many of
these puzzles are quite artificial, but you'll find yourself thinking
much more about bits in working your way through them.

\section{Logistics}

This is an individual project. All handins are electronic.
Clarifications and corrections will be posted on the course Web page.

\section{Handout Instructions}

\begin{quote}
\bf SITE-SPECIFIC: Insert a paragraph here that explains how the
instructor will hand out the {\tt datalab-handout.tar} file to the
students.
\end{quote}

Start by copying {\tt datalab-handout.tar} to a (protected) directory
on a Linux machine in which you plan to do your work.  Then give the
command
\begin{verbatim}
  unix> tar xvf datalab-handout.tar.  
\end{verbatim}
This will cause a number of files to be unpacked in the directory.
The only file you will be modifying and turning in is {\tt bits.c}.

The {\tt bits.c} file contains a skeleton for each of the 15
programming puzzles.  Your assignment is to complete each function
skeleton using only {\em straightline} code for the integer puzzles
(i.e., no loops or conditionals) and a limited number of C arithmetic
and logical operators. Specifically, you are {\em only} allowed to use
the following eight operators:
\begin{verbatim}
 ! ~ & ^ | + << >>
\end{verbatim}
A few of the functions further restrict this list.  Also, you are not
allowed to use any constants longer than 8 bits.  See the comments in
{\tt bits.c} for detailed rules and a discussion of the desired coding
style.

\section{The Puzzles}

This section describes the puzzles that you will be solving in {\tt bits.c}.

\subsection{Bit Manipulations}

Table \ref{bit-tab} describes a set of functions that manipulate and
test sets of bits.  The ``Rating'' field gives the difficulty rating
(the number of points) for the puzzle, and the ``Max ops'' field gives
the maximum number of operators you are allowed to use to implement
each function.  See the comments in {\tt bits.c} for more details on
the desired behavior of the functions.  You may also refer to the test
functions in {\tt tests.c}.  These are used as reference functions to
express the correct behavior of your functions, although they don't
satisfy the coding rules for your functions.

\begin{table}[htbp]
\begin{center}
\begin{tabular}{|l|l|c|c|}
\hline
Name & Description & Rating & Max Ops\\
\hline
{\tt bitAnd(x,y)} & \verb@x & y@ using only \verb@|@ and \verb@~@  &1&8\\
{\tt getByte(x,n)} & Get byte \verb@n@ from \verb@x@.  &2&6\\
{\tt logicalShift(x,n)} & Shift right logical. &3&20\\
{\tt bitCount(x)} & Count the number of 1's in \verb@x@.  &4&40\\
{\tt bang(x)} & Compute \verb@!n@ without using \verb@!@ operator.  &4&12\\
\hline
\end{tabular}
\end{center}
\caption{Bit-Level Manipulation Functions.} 
\label{bit-tab}
\end{table}


\subsection{Two's Complement Arithmetic}

Table \ref{two-tab} describes a set of functions that make use of the
two's complement representation of integers.  Again, refer to the
comments in {\tt bits.c} and the reference versions in {\tt tests.c}
for more information.

\begin{table}[htbp]
\begin{center}
\begin{tabular}{|l|l|c|c|}
\hline
Name & Description & Rating & Max Ops\\
\hline
{\tt tmin()} & Most negative two's complement integer & 1 &4\\
{\tt fitsBits(x,n)} & Does \verb@x@ fit in \verb@n@ bits?  &2&15\\
{\tt divpwr2(x,n)} & Compute $\texttt{x}/2^{\texttt{n}}$ & 2 & 15\\
{\tt negate(x)} & \verb@-x@ without negation & 2 & 5 \\
{\tt isPositive(x)} & {\tt x > 0}? & 3 & 8\\
{\tt isLessOrEqual(x,y)} & {\tt x <= y}? & 3 & 24\\
{\tt ilog2(x)} & Compute $\lfloor \log_2 (\texttt{x}) \rfloor$ & 4 & 90 \\
\hline
\end{tabular}
\end{center}
\caption{Arithmetic Functions}
\label{two-tab}
\end{table}


\subsection{Floating-Point Operations}

For this part of the assignment, you will implement some common
single-precision floating-point operations.  In this section, you are
allowed to use standard control structures (conditionals, loops), and
you may use both {\tt int} and {\tt unsigned} data types, including
arbitrary unsigned and integer constants.  You may not use any unions,
structs, or arrays.  Most significantly, you may not use any floating
point data types, operations, or constants.  Instead, any
floating-point operand will be passed to the function as having type
{\tt unsigned}, and any returned floating-point value will be of type
{\tt unsigned}.  Your code should perform the bit manipulations that
implement the specified floating point operations.

Table \ref{fp-tab} describes a set of functions that 
operate on the bit-level representations of floating-point numbers.
Refer to the
comments in {\tt bits.c} and the reference versions in {\tt tests.c}
for more information.

\begin{table}[htbp]
\begin{center}
\begin{tabular}{|l|l|c|c|}
\hline
Name & Description & Rating & Max Ops\\
\hline
{\verb@float_neg(uf)@} & Compute \verb@-f@ & 2 & 10\\
{\verb@float_i2f(x)@} & Compute \verb@(float) x@ & 4 & 30\\
{\verb@float_twice(uf)@} & Computer \verb@2*f@ & 4 & 30\\
\hline
\end{tabular}
\end{center}
\caption{Floating-Point Functions.  Value \texttt{f} is the floating-point number having the
same bit representation as the unsigned integer \texttt{uf}.}
\label{fp-tab}
\end{table}

Functions \verb@float_neg@ and \verb@float_twice@ must handle the full
range of possible argument values, including not-a-number (NaN) and
infinity.  The IEEE standard does not specify precisely how to handle
NaN's, and the IA32 behavior is a bit obscure.  We will follow a
convention that for any function returning a NaN value, it will return
the one with bit representation \verb@0x7FC00000@.

The included program \texttt{fshow} helps you understand the structure
of floating point numbers. To compile \texttt{fshow}, switch to the
handout directory and type:
\begin{verbatim} 
  unix> make 
\end{verbatim}
You can use \texttt{fshow} to see what an arbitrary pattern 
represents as a floating-point number:
\begin{verbatim}
  unix> ./fshow 2080374784

  Floating point value 2.658455992e+36
  Bit Representation 0x7c000000, sign = 0, exponent = f8, fraction = 000000
  Normalized.  1.0000000000 X 2^(121)
\end{verbatim}
You can also give \texttt{fshow} hexadecimal and floating point
values, and it will decipher their bit structure.

\section{Evaluation}

Your score will be computed out of a maximum of 76 points based on the
following distribution:
\begin{description}
\item[41] Correctness points.
\item[30] Performance points.
\item[5] Style points.
\end{description}


{\em Correctness points.} The 15 puzzles you must solve have been given a
difficulty rating between 1 and 4, such that their weighted sum totals
to 41.  We will evaluate your functions using the \texttt{btest}
program, which is described in the next section.  You will get full
credit for a puzzle if it passes all of the tests performed by
\texttt{btest}, and no credit otherwise.

{\em Performance points.}  Our main concern at this point in the course is
that you can get the right answer.  However, we want to instill in you
a sense of keeping things as short and simple as you can.
Furthermore, some of the puzzles can be solved by brute force, but we
want you to be more clever.  Thus, for each function we've established
a maximum number of operators that you are allowed to use for each
function. This limit is very generous and is designed only to catch
egregiously inefficient solutions.  You will receive two points for
each correct function that satisfies the operator limit.

{\em Style points.} Finally, we've reserved 5 points for a subjective
evaluation of the style of your solutions and your commenting.  Your
solutions should be as clean and straightforward as possible.  Your
comments should be informative, but they need not be extensive.

\section*{Autograding your work}

We have included some autograding tools in the handout directory ---
\texttt{btest}, \texttt{dlc}, and \texttt{driver.pl} --- to help you
check the correctness of your work.

\begin{itemize}
\item {\bf \texttt{btest:}} This program checks the functional correctness of
  the functions in {\tt bits.c}. To build and use it, type the
  following two commands:
\begin{verbatim}
  unix> make
  unix> ./btest
\end{verbatim}
Notice that you must rebuild \texttt{btest} each time you modify your {\tt
  bits.c} file. 

You'll find it helpful to work through the functions one at a time,
testing each one as you go.  You can use the {\tt -f} flag to instruct
\texttt{btest} to test only a single function:
\begin{verbatim}
  unix> ./btest -f bitAnd
\end{verbatim}
You can feed it specific function arguments
using the option flags {\tt -1}, {\tt -2}, and {\tt -3}:
\begin{verbatim}
  unix> ./btest -f bitAnd -1 7 -2 0xf
\end{verbatim}
Check the file {\tt README} for documentation on running the {\texttt
  btest} program.


\item {\bf \texttt{dlc}:} This is a modified version of an ANSI C compiler from
the MIT CILK group that you can use to check for compliance with the
coding rules for each puzzle. The typical usage is:
\begin{verbatim}
  unix> ./dlc bits.c
\end{verbatim}
The program runs silently unless it detects a problem, such as an
illegal operator, too many operators, or non-straightline code in the
integer puzzles.  Running with the \texttt{-e} switch:
\begin{verbatim}
  unix> ./dlc -e bits.c  
\end{verbatim}
causes \texttt{dlc} to print counts of the number of operators used by
each function. Type {\tt ./dlc -help} for a list of command line options. 

\item {\bf \texttt{driver.pl}:} This is a driver program that uses \texttt{btest}
and \texttt{dlc} to compute the correctness and performance points for
your solution. It takes no arguments:
\begin{verbatim}
  unix> ./driver.pl
\end{verbatim}
Your instructors will use \texttt{driver.pl} to evaluate your
solution.

\end{itemize}

\section{Handin Instructions}

\begin{quote}
\bf SITE-SPECIFIC: Insert text here that tells each student how to
hand in their {\tt bits.c} solution file at your school.
\end{quote}

\section{Advice}

\begin{itemize}

\item Don't include the \verb:<stdio.h>: header file in your {\tt
bits.c} file, as it confuses \texttt{dlc} and results in some
non-intuitive error messages. You will still be able to use {\tt
printf} in your {\tt bits.c} file for debugging without including the
\verb:<stdio.h>: header, although \texttt{gcc} will print a warning that
you can ignore.

\item The \texttt{dlc} program enforces a stricter form of C declarations
than is the case for C++ or that is enforced by \texttt{gcc}.  In
particular, any declaration must appear in a block (what you enclose
in curly braces) before any statement that is not a declaration.  For
example, it will complain about the following code:
\begin{verbatim}
  int foo(int x)
  {
    int a = x;
    a *= 3;     /* Statement that is not a declaration */
    int b = a;  /* ERROR: Declaration not allowed here */
  }
\end{verbatim}

\end{itemize}

\section{The ``Beat the Prof'' Contest}

For fun, we're offering an optional ``Beat the Prof'' contest that
allows you to compete with other students and the instructor to
develop the most efficient puzzles. The goal is to solve each Data Lab
puzzle using the fewest number of operators. Students who match or
beat the instructor's operator count for each puzzle are winners!

To submit your entry to the contest, type:
\begin{verbatim}
    unix> ./driver.pl -u ``Your Nickname''
\end{verbatim}
Nicknames are limited to 35 characters and can contain alphanumerics,
apostrophes, commas, periods, dashes, underscores, and ampersands.
You can submit as often as you like. Your most recent submission will
appear on a real-time scoreboard, identified only by your nickname.
You can view the scoreboard by pointing your browser at
\begin{verbatim}
    http://$SERVER_NAME:$REQUESTD_PORT
\end{verbatim}

\begin{quote}
\bf SITE-SPECIFIC: Replace \verb:$SERVER_NAME: and \verb:$REQUESTD_PORT:
with the values you set in the \verb:./contest/Contest.pm: file.
\end{quote}


\end{document}
